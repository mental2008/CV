% --- LaTeX CV Template - Lingyun Yang ---

% Set document class and font size
\documentclass[letterpaper, 11pt]{article}
\usepackage[utf8]{inputenc}

% Package imports
\usepackage{setspace, longtable, graphicx, hyphenat, hyperref, fancyhdr, ifthen, everypage, enumitem, amsmath, setspace, xeCJK}

% --- Page layout settings --- %

% Set page margins
\usepackage[left=1in, right=1in, bottom=0.7in, top=0.7in]{geometry}

% Set line spacing
\renewcommand{\baselinestretch}{1.15}

% --- Page formatting ---

% Set link colors
% \usepackage[dvipsnames]{xcolor}
% \hypersetup{colorlinks=true, linkcolor=RoyalBlue, urlcolor=RoyalBlue}

% Hind links
\hypersetup{hidelinks}

% Set font to Libertine, including math support
\usepackage{libertine}
\usepackage[libertine]{newtxmath}

% Remove page numbering
% \pagenumbering{gobble}

% --- Document starts here ---

\begin{document}

% Name and date of last update to this document
\noindent{\Huge{杨凌云}
\hfill{\it\footnotesize 更新于 \today}}

% --- Start the two-column table storing the main content ---

% Set spacing between columns
\setlength{\tabcolsep}{8pt}

% --- Contact information and other items --- %

\vspace{0.5cm} 
\begin{center}
\begin{tabular}{lll}
% Line 1: Email, GitHub, LinkedIn
\textbf{邮箱}: lyangbk@cse.ust.hk &
\hspace{0.25in} \textbf{GitHub}: \href{https://github.com/mental2008}{mental2008} &
\hspace{0.1in} \textbf{LinkedIn}: \href{https://www.linkedin.com/in/lingyun-yang-b2881b18a/}{lingyun-yang-b2881b18a} \\

% Line 2: Phone number, office location, website
\textbf{手机号}: (+852) 9719-0933 & 
\hspace{0.25in} \textbf{Office}: CYT 3007, HKUST & 
\hspace{0.1in} \textbf{个人网站}: \href{https://www.cse.ust.hk/~lyangbk/}{https://www.cse.ust.hk/$\sim$lyangbk/} \\ 
\end{tabular}
\end{center}

% Set the width of each column
\begin{longtable}{p{1.3in}p{4.8in}}

% --- Section: Personal profile --- %
{个人简介}
& 目前是博士二年级学生,就读于香港科技大学计算机科学与工程系 \\
& \\

% --- Section: Research interests --- %
\nohyphens{研究兴趣}
& 目前的研究兴趣主要包括云计算、大规模集群的资源管理以及运用机器学习的方法进行系统优化。 \\
& \\

% --- Section: Education --- %
{教育经历}
& \textbf{香港科技大学(HKUST)} \\
& \textit{计算机科学与工程系} \\
& 计算机科学,在读博士生 \hfill 2020 -- 至今 \\
& $\diamond$ 导师: 王威教授 \\
& \\

& \textbf{华南理工大学(SCUT)} \\
& \textit{计算机科学与工程学院} \\
& 计算机科学与技术,本科 \hfill 2016 -- 2020 \\
& $\diamond$ 就读于计算机科学与技术全英创新班 ({\it GPA: 3.82/4}) \\
& \\

% --- Section: Professional experience ---

{职业经历}
& {\textbf{阿里巴巴集团}} \hfill 中国,杭州 \\
& \textit{研究实习生}, 云原生团队 \hfill 2020 年 12 月 - 至今 \\
& $\diamond$ 导师:何剑 \\
% & $\diamond$ Built morphling - an auto-configuration framework for machine learning model serving on Kubernetes, with intelligent algorithms to quickly find the optimal configuration. \\
& \\

& {\textbf{香港科技大学}} \\
& \textit{研究助理/助教},系统算法研究小组 \hfill 2020 年 9 月 - 至今 \\
& \\
 
& {\textbf{微软亚洲研究院}} \hfill 中国,北京 \\
& \textit{研究实习生}, 创新工程组 \hfill 2019 年 7 月 - 2020 年 6 月 \\
& $\diamond$ 导师:卢乐炜,李崇 \\
& $\diamond$ Applied multiple latest attention mechanisms to face recognition models, integrated with knowledge distillation, built in CNTK. \\
& $\diamond$ Built a neural architecture search pipeline for face recognition tasks. \\
& $\diamond$ Designed a symmetric padding pooling layer with self-attention which can be easily integrated into any model structure, achieve the state-of-the-art performance and improve the robustness to various pertubations. \\
& \\

% --- Section: Awards, scholarships, etc. --- %
% --- Note: section title is spread over two lines --- %
{获奖经历}
& $\diamond$ 三好学生 \& 优秀学生干部 \hfill 2019 年 11 月 \\

& $\diamond$ 国家奖学金 \hfill 2019 年 10 月 \\

& $\diamond$ 银奖,2019 年 ICPC 中国西安程序设计竞赛(全国邀请赛) \hfill 2019 年 5 月 \\

& $\diamond$ 一等奖,第 17 届广东省程序设计竞赛 \hfill 2019 年 5 月 \\

& $\diamond$ 银奖,2019 年三七互娱杯程序设计竞赛 \hfill 2019 年 4 月 \\

& $\diamond$ 金奖,2019 年华南理工大学 ACM 程序设计竞赛 \hfill 2019 年 4 月 \\

& $\diamond$ 铜奖,2018 年 ACM-ICPC 亚洲区域赛(徐州站) \hfill 2018 年 10 月 \\

& $\diamond$ 银奖,第 1 届小米程序设计竞赛 \hfill 2018 年 9 月 \\

& $\diamond$ 金奖,2018 年华南理工大学 ACM 程序设计竞赛 \hfill 2018 年 4 月 \\

& $\diamond$ 校级一等奖学金 \hfill 2017 年 11 月 \\

& $\diamond$ 铜奖,2017 年 ACM-ICPC 亚洲区域赛(西安站) \hfill Oct. 2017 \\

& $\diamond$ 金牌,第 12 届全国青少年机器人竞赛 \hfill 2012 年 7 月\\

& $\diamond$ 冠军, 2012 年 RoboCup 青少年机器人世界杯中国区选拔赛 \hfill 2012 年 3 月 \\

% --- Section: Other interests/hobbies ---

\nohyphens{兴趣爱好}
& 喜欢阅读各种类型的书籍,打篮球,健身,美食爱好者。 \\
& 在我的业余时间,我会不定期地写博客记录一些有趣的事物。 \\
& 个人博客:\href{https://mental2008.github.io/}{https://mental2008.github.io/}. \\


% --- End of CV! --- %

\end{longtable}
\end{document}
